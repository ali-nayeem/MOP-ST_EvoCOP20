\section{Conclusion}
In this paper we introduced the problem of estimating species tree from a set of gene trees as a MOP. We showed with examples that the existing method, optimizing a single criterion, may overshoot the criterion and thus deviate from the true species tree. We selected three objectives from three existing methods. Unlike traditional MOPs, in this problem a dominated solution is often important than a non-dominated one and hence we cannot rely on the PF to contain better solutions. Therefore, we designed a specialized EMO algorithm, namely, NOSSGA, in which we embed particular traits to be able to return a tree-space containing highly accurate trees. We analyzed the difference of behavior between NOSSGA and the popular NSGAII. Finally, we found the accuracy of the best trees offered by NOSSGA is quite better three existing methods. We are currently working to find our a way to filter a limited number of better trees from the final population without the knowledge of true tree provided with the simulated dataset. Furthermore, we are improving the efficiency of NOSSGA to make it scalable for larger datasets. 