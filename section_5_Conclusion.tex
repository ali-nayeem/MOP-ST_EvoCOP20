\section{Conclusion and Future Work}
In this paper we introduced the problem of estimating species tree from a set of gene trees as a MOP. We showed with examples that the existing method, optimizing a single criterion, may overshoot the criterion and thus deviate from the true species tree. We selected three objectives from three existing methods. Unlike traditional MOPs, in this problem a dominated solution is often important than a non-dominated one and hence we cannot rely on the PF to contain highly accurate trees. Therefore, we designed a specialized EMO algorithm, namely, NOSSGA, in which we embed particular traits to be able to return a tree-space containing highly accurate trees. We analyzed the difference of behavior between NOSSGA and the popular NSGAII. Finally, we found the accuracy of the best trees offered by NOSSGA is quite better three existing methods. 

We are currently working to devise a systematic methodology to filter a limited number of better trees from the final population without the knowledge of true tree provided with the simulated dataset. At present, out mutation selects one from NNI/SPR/TBR at random with equal probability. We will improve it by adjusting the selection probabilities in an adaptive way based on the success rate of an operator in the previous generation. Also, we are planning make NNI/SPR/TBR behaving intelligently by utilizing the common patterns found across the gene trees. To make NOSSGA to process large datasets within a reasonable time, we are improving the efficiency of objective evaluations. Moreover, we are modifying another popular EMO algorithm, namely, MOEA/D, in a way to make it effective to tacke this problem.