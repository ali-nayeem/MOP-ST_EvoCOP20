\section{Conclusion and Future Work}
In this paper we have introduced the problem of estimating species tree from a set of gene trees as a MOP. We have showed with examples that the existing method, optimizing a single criterion, may overshoot the criterion and thus deviate from the true species tree. We have selected three objectives from three existing methods. Unlike traditional MOPs, optimizing an objective beyond a certain limit and discarding a dominated at an early generation can reduce the chance of generating better trees. Hence we cannot rely on the PF , sampled by a traditional EMO algorithm, to contain highly accurate trees. Therefore, we have designed a specialized EMO algorithm, namely, NOSSGA, which is a modification of NSGAII. %We have shown that, NOSSGA can generate a tree-space containing highly accurate trees. 
We have analyzed the behavioral difference between NOSSGA and NSGAII on a collection of challenging simulated datasets. We have shown that, both NOSSGA and NSGAII can generate a tree-space containing better trees than three existing methods. But the accuracy of the best trees offered by NOSSGA is much better than NSGAII.

%Finally, we found the accuracy of the best trees offered by NOSSGA is quite better three existing methods on a collection of challenging simulated datasets. 

We are currently working to devise a systematic methodology to filter a limited number of better trees from the final population without the knowledge of true tree provided with the simulated dataset. At present, out mutation selects one from NNI/SPR/TBR at random with equal probability. As future improvement, we will improve it by adjusting the selection probabilities in an adaptive way based on the success rate of an operator in the previous generation. Also, we are planning make NNI/SPR/TBR behaving intelligently by utilizing the common patterns found across the gene trees. To enable NOSSGA processing large datasets within a reasonable time, we will improving the efficiency of objective evaluations. Moreover, we are modifying another popular EMO algorithm, namely, MOEA/D, make it effective to tackling this problem.