\section{Problem Description}
\label{sec:problem}
We are given a set of $M$ rooted binary trees as a collection gene trees each having $N$-taxon. We need to construct the species tree, a rooted binary tree with $N$-taxon, from these gene trees by simultaneously optimizing the following three objective functions:  
\begin{enumerate}[label=$F_\arabic*$)]        
	\item Maximize QT 
	\subitem {\scriptsize QT: number of consistent quartets (i.e., unrooted 4-taxon tree) between the species tree and the gene trees (used by ASTRAL\cite{mirarab2014astral})}
	\item Maximize TP
	\subitem {\scriptsize TP: number of consistent triplets (i.e., rooted 3-taxon tree) between the species tree and the gene trees (used by STELAR~\cite{islam2019stelar})}
	\item Maximize PL 
	\subitem {\scriptsize PL: pseudo-likelihood estimation of the species tree utilizing the underlying triplet distribution of the gene trees (used by MP-EST~\cite{mpest})}
\end{enumerate}
 \textbf{To remain in line with the EMO literature and software frameworks, in what follows we have treated each objective as a minimization criterion}. Unlike traditional multi-objective optimization problems (MOPs), here our goal is not only to effectively sampling the Pareto front (PF) because of the following two issues:
\begin{enumerate}[label=$I_\arabic*$]
	\item \label{item:i1} Optimizing an objective beyond a certain limit can lead the resultant species tree deviate from the true tree (please see Section~\ref{subsec:observation})
	%\item 
	\item \label{item:i2} A solution that is badly dominated at an early generation can still potentially evolve into a highly accurate tree at a later generation (also reported in~\cite{qu2010multi})
\end{enumerate}

%On the other hand, PF allows degrading an objective to achieve improvement in other objectives.
%The traditional definition of convergence is may not apply to this problem. 
%Therefore in this paper, we try to design a special-purpose EMO algorithm from NSGAII by replacing which can find a tree-space containing highly accurate species trees. 
%Each of the above functions (optimized by an existing method) has some ability to predict the species tree accuracy but none of them can necessarily lead towards the most accurate species tree. Therefore, optimizing them together, by an EMO algorithm, will generate a set of candidate solutions which is expected to contain a species tree that is more accurate than the output of the existing methods.