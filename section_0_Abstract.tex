\begin{abstract}
Species tree estimation from multi-locus data is complicated as biological processes can result in different loci having different evolutionary histories. Incomplete
lineage sorting (ILS), modeled by the multi-species coalescent (MSC), is considered to be a dominant cause for gene tree incongruence. Various optimization criteria (e.g., quartet consistency, pseudo-likelihood, etc.) have been shown to be statistically consistent under the MSC model, meaning that they return the true species tree with high probability given sufficiently large numbers of accurate gene trees. However, the number of genes is limited, and estimating highly accurate gene trees is difficult. Therefore, even the statistically consistent methods may fail to reconstruct highly accurate trees under practical model conditions with limited numbers of genes and in the presence of gene tree estimation error. In this study, we present a \commentA{novel} evolutionary multi-objective optimization (EMO) algorithm (NOSSGA), a simplified version of popular NSGAII, which combines various optimization criteria to find a suitable search space containing highly accurate species trees. Our experimental results on a collection of simulated datasets demonstrate that NOSSGA can lead us to a tree-space containing significantly better trees than ASTRAL- and MP-EST-estimated trees. %\commentA{and their neighboring trees.}

%The abstract should briefly summarize the contents of the paper in 15--250 words.

\keywords{Phylogenomics  \and Multi-objective
optimization \and Evolutionary Algorithm.}
\end{abstract}